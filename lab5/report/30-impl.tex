\chapter{Практическая часть}

На рисунках \ref{img:1}-\ref{img:4} приведены файлы функций ядра на основе индивидуального задания.


\img{80mm}{1}{Не оптимизированный цикл}
\img{80mm}{2}{Конвейерная организация цикла}
\img{80mm}{3}{Частично развернутый цикл}
\img{80mm}{4}{Конвейерный и частично развернутый цикл}

\clearpage
На рисунке \ref{img:7} приведены результаты рабты приложения в режиме Emulation-SW.

\img{80mm}{7}{Результаты рабты приложения в режиме Emulation-SW}


\clearpage
На рисунке \ref{img:8} приведена копия экрана Assistant View для сборки Emulation-HW.

\img{120mm}{8}{Копия экрана Assistant View для сборки Emulation-HW}


\clearpage
На рисунках \ref{img:9}-\ref{img:10} приведены результаты работы приложения в режиме Emulation-HW.

\img{120mm}{9}{Результаты работы приложения в режиме Emulation-HW (Начало)}

\img{120mm}{10}{Результаты работы приложения в режиме Emulation-HW (Продолжение)}



\clearpage
На рисунке \ref{img:12} приведено окно внутрисхемного отладчика Vivado для сборки в режиме Emulation-HW.

\img{120mm}{12}{Окно внутрисхемного отладчика Vivado для сборки в режиме Emulation-HW}


\clearpage
На рисунке \ref{img:13} приведены результаты работы приложения в режиме Hardware.

\img{120mm}{13}{Результаты работы приложения в режиме Hardware}


\clearpage
На рисунке \ref{img:14} приведена копия экрана для вкладки «Summary».

\img{120mm}{14}{Копия экрана для вкладки «Summary»}


\clearpage
На рисунке \ref{img:16} приведена копия экрана для вкладки «System Diagram».

\img{120mm}{16}{Копия экрана для вкладки «System Diagram»}

\clearpage
На рисунке \ref{img:17} приведена копия экрана для вкладки «Platform Diagram».

\img{120mm}{17}{Копия экрана для вкладки «Platform Diagram»}


\clearpage
На рисунках \ref{img:18}-\ref{img:19} приведены копии экрана для вкладки «HLS Synthesis» для ядра сборки Hardware app\_no\_pragmas.

\img{50mm}{18}{Копии экрана для вкладки «HLS Synthesis» для ядра сборки Hardware app\_no\_pragmas (Начало)}

\img{50mm}{19}{Копии экрана для вкладки «HLS Synthesis» для ядра сборки Hardware app\_no\_pragmas (Продолжение)}



\clearpage
На рисунках \ref{img:20}-\ref{img:21} приведены копии экрана для вкладки «HLS Synthesis» для ядра сборки Hardware app\_pipelined.

\img{70mm}{20}{Копии экрана для вкладки «HLS Synthesis» для ядра сборки Hardware app\_pipelined (Начало)}

\img{70mm}{21}{Копии экрана для вкладки «HLS Synthesis» для ядра сборки Hardware app\_pipelined (Продолжение)}


\clearpage
На рисунках \ref{img:22}-\ref{img:23} приведены копии экрана для вкладки «HLS Synthesis» для ядра сборки Hardware app\_unrolled.

\img{100mm}{24}{Копии экрана для вкладки «HLS Synthesis» для ядра сборки Hardware app\_unrolled (Начало)}

\img{100mm}{25}{Копии экрана для вкладки «HLS Synthesis» для ядра сборки Hardware app\_unrolled (Продолжение)}


\clearpage
На рисунках \ref{img:24}-\ref{img:25} приведены копии экрана для вкладки «HLS Synthesis» для ядра сборки Hardware app\_pipe\_unroll.

\img{90mm}{22}{Копии экрана для вкладки «HLS Synthesis» для ядра сборки Hardware app\_pipe\_unroll (Начало)}

\img{90mm}{23}{Копии экрана для вкладки «HLS Synthesis» для ядра сборки Hardware app\_pipe\_unroll (Продолжение)}


Из рисунка \ref{img:13} можно сделать вывод о том, что наибольшее время выполнения, как и ожидалось, у цикла без оптимизаций.

Далее идет частично развернутый цикл, так как в нем все развернутые итерации выполняются параллельно и количество итераций уменьшается.

Наименьшее время выполнения было достигнуто при конвейерной обработке цикла, так как внутри цикла не оказалось зависимости по данным и цикл имел возможность начинать последующие итерации менее чем за три такта.

Одновременное применение конвейеризации и частичного развертывания тела цикла позволило ускорить обработку по сравнению с применением только развертывания тела, однако уступило по времени организации, при которой использовалась только конвейерная организация. Это, вероятно, вызвано неудачно подобранными параметрами развертывания, что вызвало замедление загрузки данных из памяти.



