\chapter*{Функциональная схема разрабатываемой системы на кристалле}

Функциональная схема разрабатываемой системы на кристалле представлена на рисунке \ref{img:scheme}.

\img{90mm}{scheme}{Функциональная схема разрабатываемой системы на кристалле}

Система на кристалле состоит из следующих блоков.

\begin{enumerate}
\item Микропроцессорное ядро Nios II/e выполняет функции управления системой.
\item Внутренняя оперативная память СНК, используемая для хранения программы
управления и данных.
\item Системная шина Avalon обеспечивает связность всех компонентов системы.
\item Блок синхронизации и сброса обеспечивает обработку входных сигналов сброса и
синхронизации и распределение их в системе. Внутренний сигнал сброса
синхронизирован и имеет необходимую для системы длительность.
\item Блок идентификации версии проекта обеспечивает хранение и выдачу уникального
идентификатора версии, который используется программой управления при
инициализации системы.
\item Контроллер UART обеспечивает прием и передачу информации по интерфейсу RS232.
\end{enumerate}

\clearpage

\chapter*{Маршрут проектирования}

\section*{Модуль в QSYS}

Для создания нового модуля системы на кристалле QSYS выполнены следующие действия.

\begin{enumerate}
	\item Создан новый модуль СНК.	
	\item Установлена частота внешнего сигнала синхронизации 50 000 000 Гц.
	\item Добавлен в проект модуль  синтезируемого миркропроцессорного ядра Nios2.
	\item Добавлен в проект модуль ОЗУ программ и данных.
	\item Добавлены компоненты Avalon System ID, Avalon UART.
	\item Создана сеть синхронизации и сбоса системы.
	\item Все блоки подключены к системной шине Avalon.
	\item Сигналы TX и RX экспортированы во внешние порты.
	\item Соединины выход IRQ блока UART c входом IRQ процессора.
	\item Выполнена настройка таблицы прерываний процессора.
	\item Назначены базовые адреса устройств.
\end{enumerate}

Результат выполненных действий приведен на рисунке \ref{img:p1}.

\img{90mm}{p1}{Модуль в Qsys}
\clearpage

\section*{Назначение портам проекта контактов микросхемы} 

Были назначены контакты в соответствии с таблицей 1 из методических указаний, а затем выполнен синтез проекта.

\img{50mm}{table}{}
\clearpage

Результат выполненных действий приведен на рисунке \ref{img:p2}.

\img{100mm}{p2}{Модуль Pin Planner}
\clearpage

\section*{Код программы микропроцессорного ядра NIOS2}


В файл  $hello\_world\_small.c$ был добавлен код эхо-программы приема-передачи по интерфейсу RS232, что показано на рисунке \ref{img:p3}.

\img{100mm}{p3}{Код программы микропроцессорного ядра NIOS2}
\clearpage

\section*{Результаты тестирования PSoC на отладочной плате}

К ПК была подключена отладочная плата с ПЛИС EPC2C20, выполнена верификация проекта с использованием программы терминала. Доработан код проекта с использованием необходимых библиотек (представлены ниже).

 $\#include "system.h"$
 
 $\#include "altera\_avalon\_sysid\_qsys.h"$
 
 $\#include "altera\_avalon\_sysid\_qsys\_regs.h"$
 
 
Доработанный код проекта, а также вывод сообщения с номером группы (52) представлены на рисунке \ref{img:p4}.

\img{100mm}{p4}{Результаты тестирования PSoC на отладочной плате}
\clearpage